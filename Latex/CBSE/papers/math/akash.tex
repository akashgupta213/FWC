\documentclass[12pt]{article}
\usepackage{graphicx}
\usepackage{amsmath}
\def\inputGnumericTable{}
\usepackage[latin1]{inputenc}

\usepackage{fullpage}
\usepackage{color}
\usepackage{array}
\usepackage{longtable}
\usepackage{calc}
\usepackage{multirow}
\usepackage{hhline}
\usepackage{ifthen}
\newcommand{\solution}{\noindent\textbf{Solution: }}
\let\vec\mathbf
\begin{document}
\noindent
\begin{minipage}{0.2\textwidth}
\includegraphics[width=\linewidth]{iiitb_comet_logo.jpg}
\end{minipage}
\hfill
\begin{minipage}{0.8\textwidth}
\raggedleft
\textbf{Name:} AKASH CHANDRA GUPTA\\
\textbf{COMET ID:} COMETFWC061\\
\textbf{email :} akashchandragupta16@gmail.com\\
\end{minipage}

\vspace{0.5cm}
\hrule
\vspace{0.5cm}

\begin{center}
\textbf{\large CHAPTER 9}\\
\textbf{\large SOME APPLICATIONS OF TRIGONOMETRY}
\end{center}


\vspace{4pt}
\section*{9.1 Introduction}

In the previous chapter, you have studied about trigonometric ratios. In this chapter,
you will be studying about some ways in which trigonometry is used in the life around
you. Trigonometry is one of the most ancient subjects studied by scholars all over the
world. As we have said in Chapter 8, trigonometry was invented because its need
arose in astronomy. Since then the astronomers have used it, for instance, to calculate
distances from the Earth to the planets and stars. Trigonometry is also used in geography
and in navigation. The knowledge of trigonometry is used to construct maps, determine
the position of an island in relation to the longitudes and latitudes.

\begin{figure}[h]
\centering
\includegraphics[width=0.2\textwidth]{a.png}
\caption{A Theodolite}
\end{figure}



\section*{9.2 Heights and Distances}

Let us consider Fig. 8.1 of prvious chapter, which is redrawn below in Fig. 9.1.
\begin{figure}[h]
\centering
\includegraphics[width=0.4\textwidth]{b.png}
\caption{Line of sight and angle of elevation}
\end{figure}


In this figure, the line AC drawn from the eye of the student to the top of the
minar is called the line of sight. The student is looking at the top of the minar. The
angle BAC, so formed by the line of sight with the horizontal, is called the angle of
elevation of the top of the minar from the eye of the student.
Thus, the line of sight is the line drawn from the eye of an observer to the point
in the object viewed by the observer. The angle of elevation of the point viewed is
the angle formed by the line of sight with the horizontal when the point being viewed is
above the horizontal level, i.e., the case when we raise our head to look at the object
(see Fig. 9.2).


\begin{figure}[h]
\centering
\includegraphics[width=0.5\textwidth]{c.png}
\caption{Angle of elevation}
\end{figure}

Now, consider the situation given in Fig. 8.2. The girl sitting on the balcony is
looking down at a flower pot placed on a stair of the temple. In this case, the line of
sight is below the horizontal level. The angle so formed by the line of sight with the
horizontal is called the angle of depression.
Thus, the angle of depression of a point on the object being viewed is the angle
formed by the line of sight with the horizontal when the point is below the horizontal
level, i.e., the case when we lower our head to look at the point being viewed
(see Fig. 9.3).
\begin{figure}[h]
\centering
\includegraphics[width=0.5\textwidth]{f.jpg}
\caption{Angle of elevation}
\end{figure}


Now, you may identify the lines of sight, and the angles so formed in Fig. 8.3.
Are they angles of elevation or angles of depression?
Let us refer to Fig. 9.1 again. If you want to find the height CD of the minar
without actually measuring it, what information do you need? You would need to know
the following:
(i) the distance DE at which the student is standing from the foot of the minar
(ii) the angle of elevation, $\angle$ BAC, of the top of the minar
(iii) the height AE of the student.
Assuming that the above three conditions are known, how can we determine the
height of the minar?
In the figure, CD = CB + BD. Here, BD = AE, which is the height of the student.

which of the trigonometric ratios can we use? Which one of them has the two values
that we have and the one we need to determine? Our search narrows down to using
either tan A or cot A, as these ratios involve AB and BC.



\section*{Example 4}

From a point $P$ on the ground the angle of elevation of the top of a $10$ m tall building
is $30^\circ$. A flag is hoisted at the top of the building and the angle of elevation of the
top of the flagstaff from $P$ is $45^\circ$. Find the length of the flagstaff and the distance
of the building from the point $P$. (You may take $\sqrt3 = 1.732$)

\solution

In Fig. 9.7, $AB$ denotes the height of the building, $BD$ the flagstaff and $P$ the given
point. Note that there are two right triangles $PAB$ and $PAD$. We are required to find
the length of the flagstaff, i.e., $DB$ and the distance of the building from the point $P$,
i.e., $PA$.

Since we know the height of the building $AB$, we will first consider the right
$\triangle PAB$.

\[
\tan 30^\circ = \frac{AB}{AP}
\]

i.e.,

\[
\frac{1}{\sqrt3} = \frac{10}{AP}
\]

Therefore,

\[
AP = 10\sqrt3
\]

i.e., the distance of the building from $P$ is $10\sqrt3$ m $= 17.32$ m.

Next, let us suppose $DB = x$ m. Then $AD = (10 + x)$ m.

Now, in right $\triangle PAD$,

\[
\tan 45^\circ = \frac{AD}{AP} = \frac{10 + x}{10\sqrt3}
\]

Therefore,

\[
1 = \frac{10 + x}{10\sqrt3}
\]

\[
x = 10(\sqrt3 - 1)
\]

So, the length of the flagstaff is $7.32$ m.

\begin{figure}[h]
\centering
\includegraphics[width=0.45\textwidth]{d.png}
\caption{Example 4 diagram}
\end{figure}




\section*{Example 5}
The shadow of a tower standing on a level ground is found to be  $40$  m longer
when the Sun's altitude is  $30$ than when it is
 $30$. Find the height of the tower.


\solution
\begin{figure}[h]
\centering
\includegraphics[width=0.4\textwidth]{e.png}
\caption{Example 5 diagram}
\end{figure}

Let the height of the tower be $h$ and the shorter shadow be $x$.

\[
\tan 60^\circ = \frac{h}{x}
\Rightarrow h = x\sqrt3
\]

\[
\tan 30^\circ = \frac{h}{x + 40}
\]

Substituting,

\[
(x\sqrt3)\sqrt3 = x + 40
\Rightarrow 3x = x + 40
\Rightarrow x = 20
\]

\[
h = 20\sqrt3
\]





\section*{9.3 Summary}

\begin{enumerate}
\item The line of sight is the line drawn from the eye of an observer to the point in the
object viewed by the observer.
with the horizontal when it is above the horizontal level, i.e., the case when we raise
our head to look at the object.
\item The angle of depression of an object viewed, is the angle formed by the line of sight
with the horizontal when it is below the horizontal level, i.e., the case when we lower
our head to look at the object.
\item The height or length of an object or the distance between two distant objects can be
determined with the help of trigonometric ratios.
\end{enumerate}

\end{document}
